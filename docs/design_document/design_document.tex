\documentclass[11pt]{article}
\usepackage[utf8]{inputenc}
\usepackage{parskip}
\usepackage[a4paper, margin=1in]{geometry}
\usepackage{graphicx}
\usepackage{hyperref}
\usepackage{listings}
\usepackage{array}
\usepackage[capitalise,noabbrev]{cleveref}
\usepackage{rotating}

% Custom json syntax highlightning
\usepackage{bera}
\usepackage{xcolor}

\definecolor{background}{HTML}{EEEEEE}
 
\lstdefinestyle{mystyle}{
	basicstyle=\ttfamily\small,
	backgroundcolor=\color{background},
	breakatwhitespace=false,        
    frame=lines,   
    breaklines=true,                 
    captionpos=b,                    
    keepspaces=true,                 
    numbers=left,                    
    numbersep=5pt,                  
    showspaces=false,                
    showstringspaces=false,
    showtabs=false,                  
	tabsize=4,
	numbers=left,
	numberstyle=\tiny\color{black},
	numbersep=10pt,
	framexleftmargin=5pt,
	framexrightmargin=5pt,
}

\lstset{style=mystyle}

\definecolor{jsonstring}{RGB}{61,107,42}
\colorlet{jsonpunct}{red!60!black}
\definecolor{jsondelim}{RGB}{20,105,176}
\colorlet{jsonnumb}{magenta!60!black}

\lstdefinelanguage{json}{
	stringstyle={\color{jsonstring}},
	string = [d]{"},
    literate=
     *{0}{{{\color{jsonnumb}0}}}{1}
      {1}{{{\color{jsonnumb}1}}}{1}
      {2}{{{\color{jsonnumb}2}}}{1}
      {3}{{{\color{jsonnumb}3}}}{1}
      {4}{{{\color{jsonnumb}4}}}{1}
      {5}{{{\color{jsonnumb}5}}}{1}
      {6}{{{\color{jsonnumb}6}}}{1}
      {7}{{{\color{jsonnumb}7}}}{1}
      {8}{{{\color{jsonnumb}8}}}{1}
      {9}{{{\color{jsonnumb}9}}}{1}
      {:}{{{\color{jsonpunct}{:}}}}{1}
      {,}{{{\color{jsonpunct}{,}}}}{1}
      {\{}{{{\color{jsondelim}{\{}}}}{1}
      {\}}{{{\color{jsondelim}{\}}}}}{1}
      {[}{{{\color{jsondelim}{[}}}}{1}
      {]}{{{\color{jsondelim}{]}}}}{1},
}

\definecolor{javared}{rgb}{0.6,0,0} % for strings
\definecolor{javagreen}{rgb}{0.25,0.5,0.35} % comments
\definecolor{javapurple}{rgb}{0.5,0,0.35} % keywords
\definecolor{javadocblue}{rgb}{0.25,0.35,0.75} % javadoc
\definecolor{javablue}{HTML}{348B98}
 
\lstset{language=Java,
	keywordstyle=\color{javapurple}\bfseries,
	stringstyle=\color{javared},
	commentstyle=\color{javagreen},
	morecomment=[s][\color{javadocblue}]{/**}{*/},
	emph=[1]% Java Classes
    {%
		AggregateIterable,
		FindIterable,
        Document,
        Arrays,
        Aggregates,
        Accumulators,
        Bson,
        RuntimeException,
        User,
        Session,
        Movie,
        Adapter,
        Sorts,
        Projections,
    },
	emphstyle=[1]{\color{javablue}},
}

\renewcommand{\arraystretch}{1.5}

\title{Task 3 -- Movie Database\\ 
	\Large Design Document}
\date{\today}
\author{Federico Fregosi, Mirko Laruina,\\
        Riccardo Mancini, Gianmarco Petrelli}

\begin{document}
\pagenumbering{gobble}
\maketitle
\vfill
\setcounter{tocdepth}{2}
\tableofcontents
\vfill
\clearpage
\setcounter{page}{1}
\pagenumbering{arabic}

\section{Specifications}

\subsection{Existing Application Summary}
The application is an aggregator of movies and movie ratings with the purpose 
of providing logged users statistics and informations about a large set of movies.
Logged-in user can also rate movies they have watched while not logged-in users 
may still use the service to browse movie rankings and statistics but they are not
able to give their rate. Only movies released in Italy are considered.

All users can search a movie and view its details (e.g., title, original title, duration, 
cast, ...) along with its average rating from users and from external sources. 

In addition, all users can browse the list of movies sorting and filtering it by many parameters
(e.g. year, genre, country, actors, ...).

System administrators can view all user profile pages and ban users. In order to do that, he can 
check the full history of ratings. Once a user is banned, he can no longer log in and his username and email cannot be used by new users.

The movie database will be built upon the publicly available IMDb dataset.

The ratings will be gathered also by periodically scraping external websites 
(e.g., Rotten Tomatoes, Coming Soon, MyMovies).

\subsection{Task 3 additions overview}
Upon the application described above, which we have already designed and 
implemented, we will add the management of followed/following relationships 
between users and the suggestion of movies to the user based on his ratings.

In particular, every logged user can follow any other user and see which movies
the latter rates in a dedicated section. In his profile a user will see a new 
section from which he can search new users to follow, browse his followed users
and the users that are following him and receive suggestions for new user to 
follow based on his curring followed and following users. In addition, through 
another session of the system, the user will be able to browse the latest 
ratings from the users he follows.

Furthermore, in his homepage, a logged user will receive suggestions about 
movies that he might like based on his ratings.

\subsection{Actors}
Anonymous user, registered user, administrator and updater ``bot''.

\subsection{Requirement Analysis}

\subsubsection{ Functional Requirements}
A \textbf{registered user} can:
\begin{enumerate}
	\item follow other registered users
	\item unfollow followed users
	\item browse the users he's following
	\item browse the users that are following him
	\item browse suggestions for new users to follow
	\item browse a list of suggested movies
\end{enumerate}

\subsubsection{Non-Functional Requirements}
\begin{itemize}
	\item \textbf{Availability}: the Database must be replicated in order to be always available.
	Write operations on the Database can be eventually consistent.
	\item \textbf{Scalability}: the application must be able to scale to an arbitrary number of servers.
	\item \textbf{Security}: passwards must be stored in a secure way.
	\item \textbf{Responsive UI}: Client-side application must provide a responsive view both for pc, 
	laptops and mobile devices.
\end{itemize}

\section{Design}

\subsection{Use-case diagram}

\begin{sidewaysfigure}[h!]
    \centering
    \includegraphics[height=18cm]{figs/use_case.pdf}
    \caption{Use-case diagram}
    \label{fig:usecase}
\end{sidewaysfigure}

The use-case diagram is shown in \cref{fig:usecase}. Different colors are used to highlight cases that are exclusive of some actors: white cases are referred to all users; blue anf green cases are referred to registered user and admin; yellow cases are exclusive of the admin. Green cases highlight the ones that 
were introduced in task 3.

\subsection{Class diagram}

\begin{figure}[h!]
    \centering
    \includegraphics[width=\textwidth]{figs/class_diagram.pdf}
    \caption{Class diagram for the identified entities}
    \label{fig:class_diagram}
\end{figure}

The class diagram is shown in \cref{fig:class_diagram}. It was decided to show \emph{Country}, \emph{Year} and \emph{Genre} as separate entities as they are some of the fields over which aggregate statistics are calculated.

% ------------------------------------------------------------------------------

\clearpage
\subsection{Data model}
\begin{figure}[h!]
    \centering
    \includegraphics[width=.7\textwidth]{figs/graph_example.pdf}
    \caption{Example graph}
    \label{fig:graph_example}
\end{figure}
The graph database will have 2 different types of nodes and 2 types of edges:
\paragraph{Node types}
\begin{itemize}
	\item \textbf{Movie}: it will contain all information needed to show a list
		of suggested movies: \emph{id}, \emph{title}, \emph{year}, 
		\emph{poster\_url}.
	\item \textbf{User}: it will contain only the username since no other 
		information is required.
\end{itemize}
\paragraph{Edge types}
\begin{itemize}
	\item \textbf{Rating} (User->Movie): it will contain the rating as 		
		attribute.
	\item \textbf{Follow} (User->User): no attribute is required.
\end{itemize}
In each following subsection, an example document is shown for every collection.

\subsection{Graph operations}
\paragraph{Movie suggestion}
Movie will be suggested based on the ratings that users that liked similar 
movies as the user. For example, in figure \ref{fig:graph_example}, 
\emph{Movie 3} may be suggested to \emph{User 1} since \emph{User 2} liked 
\emph{Movie 2} as \emph{User 1} and he also liked \emph{Movie 3}. 
This is a very simple example, in practice there 
will be multiple 
``User->liked Movies->Users that liked those movies->other liked movies ''
paths starting from the user that will be aggregated on the last movie
in the chain. In other words,
the movies suggested to the user are the ones that most users that liked the 
movies that the user likes like. Of course, this approach would not scale over 
a very big database, therefore, only the most recent ratings will be considered.
This makes the suggestions aware of the current trends, which is also a 
feature. 

\paragraph{User suggestions}
Users to follow will be suggested based on the number of common follow 
relationships. For example, in figure \ref{fig:graph_example}, \emph{User 3}
may be suggested to \emph{User 2} since \emph{User 2} follows \emph{User 1}
that follows \emph{User 3}. In practice, all ``user-follow->user-follow->user''
paths starting from the user will be considered and the users with most common
followers will be returned.

\subsection{Software Architecture}

The application will be made of the following 4 components:
\begin{itemize}
	\item \textbf{Mongo DB}: a MongoDB cluster will be deployed with
		replication.
	\item \textbf{NodeJs}: an auxiliary graph database will be used to 	
		calculate more efficiently the movie suggestions and to store 
		information about follow relationships.
	\item \textbf{React Frontend}: web-based UI.
	\item \textbf{Java Backend}: using \emph{Spring}, the \emph{Java backend} will provide REST APIs to the \emph{React frontent}.
	\item \textbf{Updater ``bot''}: two Python scripts are needed in order to nightly update the DB with the latest movies and ratings: 
	\begin{enumerate}
		\item the \textbf{IMDB parser} periodically parses the IMDb dataset to add the latest movies.
		\item the \textbf{scraper} continuously parses the rating sources to update the ratings.
	\end{enumerate}
	These scripts will executes asynchronously from the \emph{Java backend}.
\end{itemize}
\end{document}